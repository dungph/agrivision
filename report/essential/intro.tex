
\chapter*{\hfill LỜI NÓI ĐẦU \hfill}
\phantomsection
\addcontentsline{toc}{chapter}{LỜI NÓI ĐẦU}

Nấm là một loại thực phẩm phổ biến, giàu dinh dưỡng và tốt cho sức khỏe nên được sử dụng trong rất nhiều món ăn. Tại Việt Nam, nhiều trang trại nấm đã được hình thành và mang lại hiệu quả kinh tế cao cho người nông dân nhưng việc sản xuất còn mang tính thủ công, phụ thuộc nhiều vào kinh nghiệm, thời tiết dẫn đến sản lượng, chất lượng nấm không ổn định.

Ngày nay, nhờ sự phát triển vượt bậc của công nghệ IOT, công nghệ tự động hóa mà công việc của người nông dân đã được giảm nhẹ đi rất nhiều. Tuy vậy, tự động hóa trong sản xuất nấm còn chưa cao khi các thiết bị không thể hoạt động tự động theo trạng thái sinh trưởng của nấm mà cần sự điều khiển trực tiếp từ con người.

Để có thể tăng sản lượng cũng như ổn định chất lượng, quá trình sinh trưởng của nấm cần được theo dõi liên tục để theo dõi và thực hiện những tác vụ cần thiết. Việc theo dõi quá trình phát triển của nấm có thể thực hiện hiệu quả bằng công nghệ thị giác máy tính sử dụng các kỹ thuật xử lý ảnh, kỹ thuật phát hiện vật thể hay phân loại ảnh.

Mạng nơ-ron tích chập (Convolutional Neural Network) là một trong những mô hình học sâu hiệu quả đặc biệt thích hợp cho các ứng dụng nhận diện và phân loại ảnh. Dựa trên mạng nơ-ron tích chập, nhiều mô hình phát hiện vật thể hay phân loại ảnh như YOLO, EfficientNet, v.v đã được phát triển với tỉ lệ phát hiện và độ chính xác cao. Nhận thấy những mô hình phát hiện vật thể và phân loại ảnh nêu trên phù hợp với yêu cầu theo dõi sự phát triển của nấm nên đề tài sẽ tập trung vào nghiên cứu, ứng dụng các thuật toán vào mô hình theo dõi và chăm sóc nấm tự động. Báo cáo được chia thành 3 chương chính:

\hspace{1cm}Chương 1. Tổng quan về hệ thống sản xuất nấm.

\hspace{1cm}Chương 2. Cơ sở lý thuyết về mạng nơ-ron tích chập và mô hình yolo

\hspace{1cm}Chương 3. Ứng dụng mô hình yolov8 trong hệ thống trồng nấm tự động.


