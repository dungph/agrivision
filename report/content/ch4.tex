% !TeX root = ../main.tex
\cleardoublepage
\phantomsection
\chapter*{Kết luận}
\addcontentsline{toc}{section}{KẾT LUẬN}

\section*{Kết quả đạt được}

Qua tìm hiểu và nghiên cứu các hệ thống IoT dùng trong nông nghiệp kết hợp với nghiên cứu lý thuyết về mạng nơ-ron tích chập và mô hình YOLO, tôi đã triển khai thành công hệ thống giám sát trồng nấm sử dụng thị giác máy tính với các chức năng giám sát và tưới nấm tự động.

Từ thực tiễn triển khai, hệ thống đã có thể hoạt động đúng theo mong muốn đề ra. Hệ thống đã có khả năng điều khiển máy bơm tạo độ ẩm theo lập lịch và kết quả kiểm tra trạng thái phát triển. Trong kết quả kiểm tra, hệ thống có thể phát hiện vị trí và kích thước hộp bao trong hình ảnh truyền từ xa.

Ngoài ra, trong quá trình triển khai hệ thống, tôi cũng thu được một số kỹ năng hỗ trợ đắc lực cho con đường học tập và làm việc như:

\begin{itemize}
	\item Làm chủ lý thuyết và ứng dụng của CNN cũng như mô hình YOLO trong bài toán phát hiện vật thể.	
	\item Có được kỹ năng phân tích thiết kế và lập trình hệ thống nhúng.
	\item Kỹ năng sử dụng trình soạn thảo Latex trong việc viết báo cáo và tài liệu kỹ thuật.
\end{itemize}

\section*{Hạn chế}

Bên cạnh những kết quả đã đạt được, quá trình phát triển và triển khai hệ thống còn gặp một số hạn chế.

Đầu tiên, mặc dù hệ thống đã có thể hoạt động tương đối chính xác, các chức năng hiện tại của hệ thống còn chưa thực sự hoàn thiện. Vì vậy, trong tương lai cần hoàn thiện các chức năng đã có và cập nhật thêm các chức năng bổ sung.

Tiếp theo, do hệ thống chạy trên phần cứng với khả năng tính toán thấp nên hệ thống phản hồi còn chậm. Để có thể triển khai hệ thống một cách hoàn chỉnh, hệ thống hiện tại cần được tối ưu hóa về cả phần mềm và phần cứng để có thời gian phản hồi nhanh hơn.

Ngoài ra, do còn thiếu kỹ năng chăm sóc nấm, số lượng nấm thu được trong quá trình phát triển còn ít, khiến số lượng hình ảnh huấn luyện chưa đạt được con số kỳ vọng. Trong tương lai tập dữ liệu cũng cần được cập nhật nhiều loại nấm khác nhau với số lượng lớn.

\section*{Phương hướng phát triển}

Tiềm năng phát triển của IoT trong nông nghiệp còn rất lớn nhờ vào sự phát triển của công nghệ tự động hóa bằng robot, công nghệ thị giác máy tính và hệ thống phân tích dữ liệu lớn. Vì vậy, dự án có thể phát triển theo một số hướng như sau:
\begin{itemize}
	\item Phát triển hệ thống tự nhận dạng nấm mới trong ảnh mà không cần người dùng gán nhãn sử dụng các mô hình mới mạnh mẽ hơn.
	\item Phát triển hệ thống robot tự động như thu hoạch, chăm sóc giúp khép kín quy trình sản xuất nấm.
\end{itemize}







